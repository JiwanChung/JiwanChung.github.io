% ========================================================
% This document is a customizable CV/Resume template built using LaTeX.
% The template is designed for easy customization and clear structure.
%
% Author: Matthew DeVerna (www.matthewdeverna.com)
% Date: 2024
% Design: Cased on Hause Lin's CV (hauselin.com)
% 
% Project Overview:
% -----------------
% This LaTeX document is designed to help you create a professional CV or 
% resume with ease. It uses as little fancy LaTex functionality or custom functions as possible to maximize its longterm durability and flexibility.
% The document is structured into multiple sections, each loaded from 
% separate subfiles for modularity and ease of maintenance. 
%
% Key Features:
% -------------
% - Customizable sections: Education, Research Experience, Awards, Publications, etc.
% - Bookmarks in the PDF for easy navigation
% - Styled bibliography with BibLaTeX
% - Hyperlinked email and website
% - FontAwesome icons for additional styling
%
% Getting Started:
% ----------------
% 1. Customize your personal information by modifying the \mytitle command.
% 2. Add your content to the respective subfiles (e.g., education.tex, exp_research.tex).
% 3. Update the bibliography file (ref.bib) with your publications and categorize them with keywords.
%
% Important Notes:
% ----------------
% - This main file includes the overall structure and settings. 
% - Each section has its own detailed instructions for further customization.
%
% ========================================================

\documentclass[11pt]{article} % Choose the document class and font size
\usepackage[margin=1in]{geometry} % Page layout settings

% Set the citation style
\usepackage[
    % backend=biber,      % Specifies the backend to be used by BibLaTeX for processing the bibliography. 'biber' is the default backend.
    maxnames=20,        % Limits the maximum number of author names to display before abbreviating with "et al."
    style=nature,       % Sets the citation style to 'nature,' which is commonly used in scientific papers.
    sorting=none,       % Specifies the sorting order of entries in the bibliography:
                        % y - year (descending)
                        % d - descending order
                        % n - name
                        % t - title
    defernumbers=true,  % Delays the assignment of citation numbers until the end of the document, allowing for the correct order of citations within each bibliography section.
]{biblatex}
\DeclareFieldFormat{labelnumber}{\null} % Makes the numbers invisible
\renewbibmacro*{name:andothers}{% Disable "and others" truncation
  \ifboolexpr{ test {\ifnumcomp{\value{listcount}}{>}{1}}}{}{}
}
%\DeclareNameFormat{author}{%
%  \ifgiveninits
%    {\usebibmacro{name:given-family}{\namepartfamily}{\namepartgiven}{\namepartprefix}{\namepartsuffix}}
%    {\usebibmacro{name:given-family}{\namepartfamily}{\namepartgiven}{\namepartprefix}{\namepartsuffix}}%
%}
\DeclareNameFormat{author}{
  \ifboolexpr{%
    (test {\ifdefstring{\namepartfamily}{Chung}} and test {\ifdefstring{\namepartgiven}{Jiwan}})
    or (test {\ifdefstring{\namepartfamily}{Chung$\dagger$}} and test {\ifdefstring{\namepartgiven}{Jiwan}})
  }
  {% 
    \textbf{\usebibmacro{name:given-family}{\namepartfamily}{\namepartgiven}{\namepartprefix}{\namepartsuffix}}%
  }
  {% 
    \usebibmacro{name:given-family}{\namepartfamily}{\namepartgiven}{\namepartprefix}{\namepartsuffix}%
  }%
}

%\addbibresource{ref.bib} % Adds the bibliography resource file 'ref.bib' containing all the references.

% Allows columns that stretch across pages
\usepackage{longtable}

% Table functionality and beautification (not strictly needed)
\usepackage{bookmark}

% Use icons, if you want.
% All available icons: http://mirrors.ibiblio.org/CTAN/fonts/fontawesome5/doc/fontawesome5.pdf
\usepackage{fontawesome}

% Allows font justification control (needed for clean pub-list formatting)
\usepackage{ragged2e}

% For underlining with line breaks
\usepackage{soul} 

% All fonts: https://tug.org/FontCatalogue/
\usepackage{kpfonts} % More professional font
% \usepackage[default]{sourcecodepro} % Code-like font
\usepackage[T1]{fontenc}

% Control hyperlinks and colors
% CUSTOM COLORS INCLUDED DIRECTLY AFTER \begin{document}
\usepackage{xcolor}
\usepackage{hyperref}
\hypersetup{
    colorlinks=true,        % Enable colored links
    breaklinks=true,        % Allow links to break across lines
    linkcolor=cornflowerblue,    % Color of internal links
    urlcolor=cornflowerblue,     % Color of URL links
    anchorcolor=cornflowerblue,  % Color of anchors
    citecolor=cornflowerblue,    % Color of citations
    pdftitle={Your Title},    % Title of the PDF
    pdfauthor={Your Name}, % Author of the PDF
    bookmarksopen=true,      % Open bookmarks panel at start
}

%%% CONVENIENCE FUNCTIONS GO HERE %%%
%%% ----------------------------- %%%
\newcommand{\mytitle}[4]{
  \begin{center}
    \Large\textbf{#1}\normalsize \\ % Name in large bold font
    \href{mailto:#2}{#2} \\ % Email with mailto: link
    \href{https://#3}{#3} \\ % Website with link
    #4 % Address
  \end{center}
}
%%% ----------------------------- %%%


\DeclareBibliographyDriver{inproceedings}{%
  \textbf{\printfield{title}}\newline % Paper title
  \printnames{author}\newline % Authors
  \printfield{booktitle}\space % Conference Name
  \printfield{year}% Year
}

\DeclareBibliographyDriver{article}{%
  \textbf{\printfield{title}}\newline % Paper title
  \printnames{author}\newline % Authors
  \printfield{journaltitle}\space % Conference Name
  \printfield{year}% Year
}


\begin{document}
\include{colors.tex} % Load custom colors from colors file
\mytitle{Jiwan Chung}{jiwan.chung.research@gmail.com}{jiwanchung.github.io} % Insert your custom title


% Ensure right side margin is not surpassed by bibliography and the right margin is aligned throughout
\RaggedRight

\pdfbookmark[1]{Research Interest}{}
\section*{Research Interest}
%My research goal is understanding how knowledge emerges and develops by replicating these processes in machines. Arguably, the most fundamental source of human knowledge lies in perception\footnote{\url{https://plato.stanford.edu/entries/epistemology/\#SourKnowJust}}, which forms the basis of how we interpret and learn from the world around us. To reflect this, my current focus is on multimodal artificial intelligence, where I explore challenges that arise from introducing multimodal inputs and outputs to large language models (LLMs). 

My research centers on multimodal understanding and generation, with a particular focus on Multimodal Large Language Models (MLLMs). I explore how multimodal context and text can be more effectively connected in AI systems, spanning both architectural innovations and evaluative methodologies.

% These \pdfbookmark lines create bookmarks in the exported PDF document that display in the left pane.
% Value in [] sets the indentation level of the bookmark
\pdfbookmark[1]{Education}{}
\section*{Education}
\vspace{-1.2em}
% Add your educational background here!

% NOTE: If you want to remove the "Expected" footnote, you will want to remove:
% - Directly below: \renewcommand, \setcounter
% - In the table: \footnotemark in the left column
% - After the table: \footnotetext, \renewcommand, \setcounter

% Different numbers in "\setcounter{footnote}{0}" use different symbols
\renewcommand{\thefootnote}{\fnsymbol{footnote}}
\setcounter{footnote}{0}


\begin{longtable}[l]{@{}p{.125\textwidth} p{0.875\textwidth}@{}}
    2023-26\footnotemark[1] & \textbf{Ph.D.}, Artificial Intelligence, Yonsei University \\
    & \textit{Advisor: Seon Joo Kim (present)}, \textit{Youngjae Yu ($\sim$ August 2025)} \\
    2019-23 & \textbf{M.S.}, Computer Engineering, Seoul National University \\
    & \textit{Advisor: Gunhee Kim} \\
    2014-18 & \textbf{B.Sc.}, Computer Science and Philosophy, Yonsei University \\
    & \textit{Double Major} \\
\end{longtable}


% Add text for the custom footnote
\footnotetext[1]{Expected.}

% Restore the default footnote numbering
\renewcommand{\thefootnote}{\arabic{footnote}}
\setcounter{footnote}{1}


\vspace{-1.5em}
\pdfbookmark[1]{Research Experience}{exp_research}
\section*{Research Experience}
\label{exp_research}
\vspace{-1.5em}
% List your research experience here

\begin{longtable}[l]{@{}p{.16\textwidth} p{0.875\textwidth}}
    2025 Summer & Research Intern, Microsoft Research AI Frontiers \\
    2024 Spring & Research Intern, LG AI Research  \\
    2023 Fall & Research Intern, Naver  \\
\end{longtable}



%\pdfbookmark[1]{Awards \& Honors}{awards}
%\section*{Awards \& Honors}
%\label{awards}
%\input{awards.tex}

%\vspace{-1.5em}
\pdfbookmark[1]{Peer-reviewed Conference Proceedings}{pubs_conf}
\section*{Conference Papers}
\label{pubs_conf}

% Add equal contribution dagger
\vspace{-.75em}
\small
\faGoogle~\href{https://scholar.google.co.kr/citations?user=l4UBOZAAAAAJ}{Google Scholar}\\
$\dagger \rightarrow$ Equal contribution
\normalsize

\begin{refsection}[ref_conf.bib]
\nocite{*}
\printbibliography[heading=none,resetnumbers=true]
\end{refsection}


\vspace{-1.5em}
\pdfbookmark[2]{Preprints}{pubs_preprint}
\section*{Preprints}
\label{pubs_preprint}

\begin{refsection}[ref_preprint.bib]
\nocite{*}
\printbibliography[heading=none,resetnumbers=true]
\end{refsection}




%\pdfbookmark[1]{Presentations}{presentations}
%\section*{Presentations}
%\label{presentations}

% Include any additional details here
% \vspace{-.75em}
% \small
% $\dagger \rightarrow$ Equal contribution
% \normalsize


%\pdfbookmark[1]{Teaching}{teaching}
%\section*{Teaching}
%\label{teaching}
%\input{teaching.tex}


%\pdfbookmark[1]{Academic Service}{service}
%\section*{Academic Service}
%\label{service}
%\input{service.tex}


% Pretty ending with the date last updated
\centering
\rule{0.25\linewidth}{0.4pt}\\
\medskip
Last updated: \today

\end{document}
